Hugo Distler hat \emph{Es ist ein Ros entsprungen} nicht in der hier vorliegenden Form, als durchkomponiertes Strophenlied, komponiert. Vielmehr sind die einzelnen Strophen leitmotivisch über sein \enquote{Oratorium mit kammermusikalischem Charakter} \emph{Die Weihnachtsgeschichte}, op.10, verteilt. In dem Werk, das von Vertonungen über die Texte \emph{Das Volk, so im Finstern wandelt} und \emph{Also hat Gott die Welt geliebt} gerahmt ist, stehen zwischen dem von Solisten ausgeschmückten Evangelienbericht 6 Choralvariationen über das Lied. Die fünf Choralvariationen, die über den Strophentext von \emph{Es ist ein Ros entsprungen} gesetzt sind, bilden die Strophen der vorliegenden Ausgabe. Eine sechste doppelchörige Variation der Melodie über den Text \enquote{Die Hirten zu der Stunden machten sich auf die Fahrt das Kindlein sie bald funden mit seiner Mutter zart. Die Engel sangen schon, sie lobten Gott, den Herren in seinem höchsten Thron} hat nicht Eingang in die vorliegende Zusammenstellung gefunden, weil dieser Text nicht zum Textmaterial von \emph{Es ist ein Ros entsprungen} gehört und in einem alleinstehenden Strophenlied auch einen inhaltlichen Bruch bedeuten würde.

Die einzelnen Choräle als Strophen eines Liedes aufzufassen reißt die Musik aus dem Kontext, in den der Komponist sie gestellt hatte. Eine Spur dieses Kontextverlusts ist der \enquote{Ei-a}-Part der Bässe in der dritten Strophe, die zudem mit \enquote{Choral, wie ein Wiegenlied} überschrieben ist. Diese beiden deutlichen Bezüge auf die Form des Wiegenlieds ergeben sich nicht aus dem Text von\emph{Es ist ein Ros entsprungen}, erklären sich aber aus der Stellung der Strophe in Opus 10: Dort folgt der Choral auf die Evangelistenerzählung Lk 2,2-2-7 \enquote{[...] und sie gebar ihren ersten Sohn. Und sie wickelte ihn in Windeln und legte ihn in eine Krippe, denn sie hatten sonst keinen Raum in der Herberge.}
Der Zusammenhang, den die vorliegende Ausgabe herstellt, ist aber nicht völlig künstlich. Natürlich rechnete Distler damit, dass sein Publikum das bekannte Lied als die Quelle der Choraltexte und der Melodie erkennt, die als Grundlage für die Choralvariationen diente. Distler hat auch den Zusammenhang der einzelnen Strophen insofern musikalisch gewahrt, als alle hier als Strophen wiedergegebenen Choräle in der gleichen Tonart - E-Dur - geschrieben sind, während die Passagen des Erzählers in Distlers Weihnachtsgeschichte und auch die oben erwähnten Rahmenchöre wechselnde Tonarten aufweisen.

Der hier vorliegende Edition liegen Distlers Autograph und die Erstausgabe von 1933 zugrunde: 
\begin{itemize}
	\item Hugo Distler: \emph{Die Weihnachtsgeschichte, Sketches, V (4), Coro, op.10, LüdD p.441}, Bayrische Staatsbibliothek, Mus.N. 119,39, [caption title:] Jes. 9, 1933 (http://daten.digitale-sammlungen.de/~db/0006/bsb00068297/images/)
	\item Hugo Distler: \emph{Die Weihnachtsgeschichte. Für vierstimmigen gemischten Kammerchor und vier Vorsänger, Op. 10}. Bärenreiter-Ausgabe 690, Bärenreiter, Kassel 1933. 
\end{itemize}

Auffällig ist, dass im Choral \emph{Lob, Ehr sei Gott, dem Vater} (entspricht Strophe 4 unseres Liedes) in den Stimmen Sopran, Tenor und Bass eine ABA-Struktur vorliegt: 

Auffälligkeiten in der Erstausgabe:
Strophe 4: 
+ Probenmarkierung (B): Im Bass ist ein vier-viertel-Takt-zeichen nicht notiert
+ ABA-Struktur in Sopran, Tenor und Bass, aber (wegen einiger weniger Noten) nicht im Alt?! Oder doch eher Druckfehler in den entsprechenden Takten im Alt? -> Vgl. mit Autograph
