\documentclass{book}

\usepackage[margin=1.5cm]{geometry}
\setlength{\parindent}{0cm} 
\usepackage[utf8]{inputenc}
\usepackage[T1]{fontenc}

\usepackage[german]{babel}

\usepackage{fontspec}
  \setmainfont{EB Garamond}

\usepackage[german=guillemets]{csquotes}
\usepackage[hidelinks]{hyperref}

\usepackage{multicol}
  \setlength{\columnsep}{1cm}

% format section, subsection and subsubsection headings
\newcounter{mysubsection}[section]
\newcounter{mysubsubsection}[mysubsection]

\usepackage{titlesec}
  \titleformat{\section}{\Huge\scshape\filcenter}{}{1em}{}
  \titleformat{\subsection}{\Large\scshape\filcenter}{\stepcounter{mysubsection} \themysubsection
}{1em}{}
  \titleformat{\subsubsection}{\large\filcenter}{\stepcounter{mysubsubsection}\themysubsection . \themysubsubsection}{1em}{}

\pagenumbering{roman}

% provide number of last page to be used by fancyhdr 
\usepackage{zref-abspage,zref-lastpage}
\makeatletter
\newcommand*{\iffancylastpage}{%
    \ifnum\zref@extractdefault{LastPage}{abspage}{-1}%
        =\numexpr\value{abspage}+1\relax
      \expandafter\@firstoftwo
    \else
      \expandafter\@secondoftwo
    \fi
  }
\makeatother
\usepackage{fancyhdr}
  \pagestyle{fancy}
  \fancyhf{} % clear header and footer
  \renewcommand{\headrulewidth}{0pt}  % remove headrule
  \fancyhead[LO, RE]{\thepage}  % book-like pagenumbering in header
  
  \fancyfoot[R]{  % Version on last page
    \iffancylastpage{Version 1.0.0}{}%
  }


% necessary things for the interlinear translation
\usepackage{stackengine}
  \setstackEOL{\cr}

\begin{document}

\section{Editorische Anmerkungen}
\begin{multicols}{2}

\subsection{Quellen}

Dem Notentext liegen zugrunde das offenbar von Distler für die
Veröffentlichunug erstellte Manuskript\footnote{Hugo Distler: \emph{Der
  Jahrkreis}, Bayerische Staatsbibliothek, Signatur: BSB Mus. N. 119,6,
  \url{https://daten.digitale-sammlungen.de/\textasciitilde db/0007/bsb00073378/images/}}
sowie die Erstausgabe\footnote{Hugo Distler: \emph{Der Jahrkreis. Eine
  Sammlung von 52 zwei- und dreistimmigen geistlichen Chormusiken zum
  Gebrauch in Kirchen-, Schul- und Laienchören.} Bärenreiter-Ausgabe
  676, Bärenreiter, Kassel 1933 {[}Erstausgabe{]}, S. 5.} von Distlers
\emph{Der Jahrkreis} (op. 5).

\subsection{Entscheidungen}

Die beiden Quellen stimmen hinsichtlich des Notentextes weitgehend
überein. Sie unterscheiden sich an den folgenden Stellen:

\begin{enumerate}
\item
  Das Manuskript schreibt im vierten Takt vor der Generalpause in allen
  Stimmen ein Atemzeichen. Dieses Atemzeichen ist im Druck nicht
  enthalten. Das Atemzeichen mag aufgrund der Generalpause technisch
  nicht notwendig erscheinen. Atemzeichen beinhalten bei Distler häufig
  jedoch nicht nur eine technische, sondern auch eine Gliederungsfunktion. Daher schreibt die vorliegende Ausgabe das Atemzeichen.
\item
  Das Manuskript schreibt dem Sopran im Takt 5 auf der vierten Zählzeit
  -- der üblichen Melodie des Liedes folgend -- das Melisma
  \texttt{es'{}8({[}\ d\textquotesingle{}8{]})} zu, während
  in der gedruckten Erstausgabe an selber Stelle eine Viertelnote
  \texttt{f'{}4} steht. Die vorliegende Ausgabe folgt der
  Variante im Erstdruck, da eine solche Abweichung von der bekannten
  Melodie nur sehr unwahrscheinlicherweise auf einen korrigierenden
  Eingriff des Herausgebers zurückgeht, sondern höchstwahrscheinlich
  eine Korrektur Distlers darstellt.
\item
  Das Melisma \texttt{f'8({[}\ es'8{]})} auf der dritten Zählzeit im Takt
  drei des Alt ist im Manuskript nur durch die Textverteilung
  ausgezeichnet; im Druck ist das Melisma zusätzlich durch einen Bogen
  und durch die Balkung der Achtelnoten ausgedrückt. Da hinsichtlich der
  tatsächlichen Textverteilung in beiden Quellen Einigkeit herrscht, und
  da die Variante des Erstdrucks den Notensatzkonventionen entspricht,
  ist diese in der vorliegenden Ausgabe übernommen worden.
\item
  Im fünften Takt des zweiten Satzes schreibt der Erstdruck in der
  Altstimme beim \texttt{b} ein Erinnerungsvorzeichen, das im Manuskript
  nicht enthalten ist. Die vorliegende Ausgabe schreibt das Vorzeichen.
\item
  Das Melisma im vorvorletzten Takt der Motette ist im Manuskript im
  Sopran wiederum lediglich durch die Textverteilung beschrieben. Im Alt
  und in der Männerstimme ist das entsprechende Melisma jedoch ebenfalls
  durch Bindebogen angezeigt. Wie schon bei 3. zeigt der Druck hier das
  Melisma auch im Sopran durch den Bindebogen an. Die vorliegende
  Ausgabe folgt hier wie oben der Erstausgabe, und aus den gleichen
  Gründen.
\item
  Der Druck teilt den Alt letzten Takt im letzten Ton auf \texttt{g} und
  \texttt{d}. Das Manuskript schreibt an dieser Stelle für den Alt nur
  \texttt{g}, sodass das Stück mit dem Intervall \texttt{g-h}, statt mit
  einem G-Dur-Dreiklang endet. Da auch dieser Unterschied eine Korrektur
  Distlers vor Drucklegung vermuten lässt, folgt die vorliegende Ausgabe
  hier dem Erstdruck.
\end{enumerate}

\subsection{Eingriffe}

\subsubsection{Der Satz der 2. Strophe}

Die Textverteilung weicht von der Textverteilung in den Quellen ab: Der
Text der zweiten Gedichtstrophe steht in den Quellen frei zwischen dem
ersten Satz, dem die erste Strophe unterlegt ist, und dem zweiten Satz,
dem die dritte Strophe unterlegt ist. Das Stück ist mit dem Hinweis
``Von dieser Motette sind alle Verse zu singen'' versehen. Distler
schreibt im Vorwort als Hinweis zur Ausführung der Motetten in \emph{Der
Jahrkreis}: ``Es ist nicht notwendig, jeweils die ganze Motette in all
ihren Verstexten und verschiedenen Sätzen durchzuführen''. Für
\emph{Maria durch ein' Dornwald ging} machte Distler also ausdrücklich
eine Ausnahme von dieser allgemeinen Regel.

Um die praktische Arbeit mit der Note zu erleichtern, wurde die zweite
Strophe unterlegt den Notentext gesetzt.

\subsubsection{Volta}

Entsprechend den Notensatzkonventionen wurde der doppelte Taktstrich am
Ende des ersten Satzes durch den Volta-Taktstrich ersetzt.

\subsubsection{Textverteilung}

Da die Metren des ersten und zweiten Gedichtstrophe verschieden sind,
wurde durch das Aussetzen der zweiten Gedichtstrophe in den Notentext
eine Entscheidung über die Textverteilung in der Sopranstimme nötig. Der
Vorschlag folgt der üblichen Textverteilung der zweiten Strophe im Lied
\emph{Maria durch ein Dornwald ging}. Die Entscheidung der vorliegenden
Ausgabe stimmt dabei auch mit den Textverteilung in den Aufnahmen der
Motette von Wolfgang Unger\footnote{Hugo Distler: Liturgische Sätze.
  Aufgenommen vom Leipziger Universitätschor und dem Pauliner
  Kammerorchester, Leitung: Wolfgang Unger, Thorofon 2001,
  \url{https://www.youtube.com/watch?v=dARDB8tENTs\&list=OLAK5uy\_mzyxKIPHbUph\_TlsI5NWk\_JBOaHwp1ISk\&index=24}}
und Christian Grube\footnote{Hugo Distler: Die Weihnachtsgeschichte. Und
  Liedmotetten zur Weihnacht. Aufgenommen vom Kammerchor der Hochschule
  der Künste Berlin, Leitung: Christian Grube, Thorofon 2008,
  \url{https://www.youtube.com/watch?v=dARDB8tENTs\&list=OLAK5uy\_mzyxKIPHbUph\_TlsI5NWk\_JBOaHwp1ISk\&index=24 }}
überein.

Da die Altstimme textlich unabhängig von der Sopranstimme gesetzt ist,
ist auch die Textverteilung der zweiten Gedichtstrophe in der Altstimme
in der Erstausgabe nicht eindeutig, sodass auch hier eine Entscheidung
notwendig wurde. Dass Distler die Textverteilung offenbar nicht selbst
ausnotiert hat, lässt vermuten, dass er eine zur ersten Strophe analoge
Textveteilung im Sinn hatte. Die vorliegende Ausgabe strebt eine enge
Analogie der Textverteilung der ersten Gedichtstrophe auf die Noten des
Alts an:

Die Wiederholung der ersten drei Silben des dritten Verses der zweiten
Strophe sowie die Beibhaltung identischer Melismen wirft keine
semantischen oder lexikalischen Probleme auf. Daher wurude diese
Textverteilung gewählt. Die vorliegende Ausgabe stimmt damit in dieser
Hinsicht mit der Unger-Aufnahme, nicht aber mit der Grube-Aufnahme
überein, die den Text der Altstimme in der ersten Strophe nicht als eine
Wiederholung des dritten, sonder des \emph{ersten} Verses der
\emph{ersten} Strophe auffasst und den Alt in der zweiten Strophe den
Text
\texttt{Ma\ -\/-\ ri\ -\/-\ a,\ was\ trug\ Ma\ -\/-\ ri\ -\/-\ a\ un\ -\/-\ ter\ ih\ -\/-\ rem\ Her\ -\/-\ zen?}
singen lässt.

Die Uneindeutigkeit, ob der Alt-Text im ersten Satz als Wiederholung des
ersten oder aber des dritten Verses aufzufassen ist, ergibt sich daraus,
dass der erste und der dritte Vers der ersten Gedichtstrophe identisch
sind. Dies trifft auf den ersten und dritten Vers der zweiten Strophe
nun gerade nicht zu, sodass sich bei der Analogie-Übertragung besagte
Uneindeutigkeit ergibt. Das von der Grube-Aufnahme angestrebte
Ausschreiben des Alt-Textes analog zum ersten Vers der ersten Strophe
steht aber vor dem Problem, dass die dann zu wiederholenden ersten drei
Silben \texttt{Was\ tru\ Ma\ -\/-} wären, und die wiederholte Phrase
damit lexikalisch durchbrochen wäre und keinen Sinnzusammenhang bilden
würde. Diese Textveteilug erschiene für Distler untypisch, insbesondere,
weil Distler in der ersten Strophe an dieser Stelle ein Atemzeichen
schreibt. Das Atemzeichen steht bei Distler in der Regel für eine
Einheit, oft eine semantisch-musikalische, mitunter eine rein
musikalische Einheit. Der Textrumpf \texttt{Was\ tru\ Ma\ -\/-} würde
dem nicht entsprechen.

Die Grube-Aufnahme begegnet diesem Problem, indem sie statt der
Wiederholung dieser Silben \texttt{Ma\ -\/-\ ri\ -\/-\ a} an die
fragliche Stelle, d.h. auf Zähzeit 6 im Takt 4 bis Zählzeit 2-und im
Takt 5 setzt und dem Alt den Text
\texttt{Ma\ -\/-\ ri\ -\/-\ a,\ was\ trug\ Ma\ -\/-\ ri\ -\/-\ a\ oh\ -\/-\ ne\ Schmer\ -\/-\ zen?}
singen lässt.

Damit, oder auch mit anderen denkbaren Berichtigungen, wird jedoch die
Analogie zur ersten Strophe aufgegeben. Will man die Leerstelle füllen,
die die Quellen in der Frage der Textverteilung der zweiten Strophe
hinterlassen, dann erscheint die in dieser Ausgabe gewählte strenger
durchführbare Analogie zur ersten Strophe als die zwanglosere Adaption.

\subsubsection{Taktnummerierung}

Die Takte sind in keiner Quelle nummeriert. Aus praktischen
Gesichtspunkten wurden die Takte in der vorliegenden Ausgabe nummeriert.
Dabei erschien es wegen der Kürze des Stücks praktikabel, die Takte
fortlaufend zu nummerieren, d.h. über die Grenze der beiden Sätze der
Motette hinweg.

\subsection{Aufnahme in das Korpus}

Das Lied \emph{Maria durch ein' Dornwald ging} ist heute ein bekanntes Adventslied. Ursprünglich war es ein Wallfahrtslied, möglicherweise auch ein Lied zum Fest Visitatio Mariae (\enquote{Mariä Heimsuchung}), das den Besuch Marias bei Elisabeth feiert, wie er in Lk Lk 1,39–56\footnote{vgl. etwa die Einheitsübersetzung, \url{https://www.bibleserver.com/EU/Lukas1\%2C39-56} } beschrieben ist. Zwar liegt der Festtag Mariä Heimsuchung nicht in der Adventszeit; der Besuch Marias bei Elisabeth ist aber wegen der Thematisierung der Schwangerschaft Marias und Elisabeths insbesondere in der Kirchenmusik stark mit der Adventszeit assoziiert; so ist etwa Johann Eccards noch eindeutiger dem Fest Mariä Heimsuchung zuzuordnendes \emph{Übers Gebirg Maria geht} fest im Advent etabliert; ebenso ist es üblich, in der Adventszeit Chorsätze des \emph{Ave Maria} und auch des \emph{Magnificat} zu singen, welche beide in Lk 1,39–56 wurzeln.

Die Motette wird auch im Inhaltsverzeichnis der Erstausgabe von \emph{Der Jahrkreis} dem Advent zugeordnet. Damit ist die Zugehörigkeit der Motette zum Korpus der in der Weihnachts- und Adventszeit gesungenen Weihnachts-, Advents- und Winterlieder unstrittig.

\end{multicols}
\section{Übersetzung und Transkription}

\begin{flushleft}

\newcommand{\ilfgerman}[1]{
    \normalsize{
      #1
    }
}

\newcommand{\ilfgrammar}[1]{
    \scriptsize{
    \textit{
      #1
    }
}
}

\newcommand{\ilfipa}[1]{
    \normalsize{
      #1
    }
}

\newcommand{\ilfenglish}[1]{
    \textit{
      #1
    }
}


\newcommand{\ilflinebreaksize}{ 6pt }
\newcommand{\ilfcolmargin}{ 2pt }


\setlength{\parindent}{0pt}
\setstackgap{L}{12pt}
\setstackEOL{\cr}

\mbox{\Longstack{
      \ilfgerman{Maria                    } \cr
      \ilfgrammar{Nom. Sgl. \enquote{Maria}} \cr
      \ilfipa{maˈʁiːa                  } \cr
      \ilfenglish{Mary                     } \cr
  }} \hspace{\ilfcolmargin}
  \mbox{\Longstack{
      \ilfgerman{durch          } \cr
      \ilfgrammar{\enquote{durch}} \cr
      \ilfipa{dʊʁç           } \cr
      \ilfenglish{through        } \cr
  }} \hspace{\ilfcolmargin}
  \mbox{\Longstack{
      \ilfgerman{ein’                   } \cr
      \ilfgrammar{Akk. Sgl. \enquote{ein}} \cr
      \ilfipa{aɪ̯n                   } \cr
      \ilfenglish{a                      } \cr
  }} \hspace{\ilfcolmargin}
  \mbox{\Longstack{
      \ilfgerman{Dornwald                    } \cr
      \ilfgrammar{Akk. Sgl. \enquote{Dornwald}} \cr
      \ilfipa{ˈdɔʁnvalt                   } \cr
      \ilfenglish{thorny wood                 } \cr
  }} \hspace{\ilfcolmargin}
  \mbox{\Longstack{
      \ilfgerman{ging,} \cr
      \ilfgrammar{3. Pers. Sgl. Prät. Ind. \enquote{gehen}} \cr
      \ilfipa{ɡɪŋ} \cr
      \ilfenglish{went} \cr
  }} \hspace{\ilfcolmargin}
    \linebreak

\vspace{\ilflinebreaksize}

\mbox{\Longstack{
      \ilfgerman{Kyrieleison!} \cr
      \ilfgrammar{\enquote{Kyrieleison}} \cr
      \ilfipa{ˌkyːʁiː ʔeˈlɛɪ̯zɔn} \cr
      \ilfenglish{Kyrieleison!} \cr
  }} \hspace{\ilfcolmargin}
    \linebreak

\vspace{\ilflinebreaksize}

\mbox{\Longstack{
      \ilfgerman{Maria                    } \cr
      \ilfgrammar{Nom. Sgl. \enquote{Maria}} \cr
      \ilfipa{maˈʁiːa                  } \cr
      \ilfenglish{Mary                     } \cr
  }} \hspace{\ilfcolmargin}
  \mbox{\Longstack{
      \ilfgerman{durch          } \cr
      \ilfgrammar{\enquote{durch}} \cr
      \ilfipa{dʊʁç           } \cr
      \ilfenglish{through        } \cr
  }} \hspace{\ilfcolmargin}
  \mbox{\Longstack{
      \ilfgerman{ein’                   } \cr
      \ilfgrammar{Akk. Sgl. \enquote{ein}} \cr
      \ilfipa{aɪ̯n                   } \cr
      \ilfenglish{a                      } \cr
  }} \hspace{\ilfcolmargin}
  \mbox{\Longstack{
      \ilfgerman{Dornwald                    } \cr
      \ilfgrammar{Akk. Sgl. \enquote{Dornwald}} \cr
      \ilfipa{ˈdɔʁnvalt                   } \cr
      \ilfenglish{thorny wood                 } \cr
  }} \hspace{\ilfcolmargin}
  \mbox{\Longstack{
      \ilfgerman{ging,} \cr
      \ilfgrammar{3. Pers. Sgl. Prät. Ind. \enquote{gehen}} \cr
      \ilfipa{ɡɪŋ} \cr
      \ilfenglish{went} \cr
  }} \hspace{\ilfcolmargin}
    \linebreak

\vspace{\ilflinebreaksize}

\mbox{\Longstack{
      \ilfgerman{der             } \cr
      \ilfgrammar{Reflexivpronomen} \cr
      \ilfipa{deːɐ̯           } \cr
      \ilfenglish{which           } \cr
  }} \hspace{\ilfcolmargin}
  \mbox{\Longstack{
      \ilfgerman{hat                               } \cr
      \ilfgrammar{3. Pers. Sgl. aux. \enquote{haben}} \cr
      \ilfipa{hat                               } \cr
      \ilfenglish{had                               } \cr
  }} \hspace{\ilfcolmargin}
  \mbox{\Longstack{
      \ilfgerman{in          } \cr
      \ilfgrammar{\enquote{in}} \cr
      \ilfipa{    ɪn      } \cr
      \ilfenglish{in          } \cr
  }} \hspace{\ilfcolmargin}
  \mbox{\Longstack{
      \ilfgerman{sieben          } \cr
      \ilfgrammar{\enquote{sieben}} \cr
      \ilfipa{    ˈziːbm̩     } \cr
      \ilfenglish{seven           } \cr
  }} \hspace{\ilfcolmargin}
  \mbox{\Longstack{
      \ilfgerman{Jahrn                  } \cr
      \ilfgrammar{Dat. Pl. \enquote{Jahr}} \cr
      \ilfipa{    jaːɐ̯n             } \cr
      \ilfenglish{years                  } \cr
  }} \hspace{\ilfcolmargin}
  \mbox{\Longstack{
      \ilfgerman{kein          } \cr
      \ilfgrammar{\enquote{kein}} \cr
      \ilfipa{kaɪ̯n         } \cr
      \ilfenglish{no            } \cr
  }} \hspace{\ilfcolmargin}
  \mbox{\Longstack{
      \ilfgerman{Laub                    } \cr
      \ilfgrammar{Akk. Sgl. \enquote{Laub}} \cr
      \ilfipa{laʊ̯p                   } \cr
      \ilfenglish{greenery                } \cr
  }} \hspace{\ilfcolmargin}
  \mbox{\Longstack{
      \ilfgerman{getrag’n} \cr
      \ilfgrammar{Part. Perf. \enquote{tragen}} \cr
      \ilfipa{ɡəˈtʁaːɡn̩ } \cr
      \ilfenglish{beared} \cr
  }} \hspace{\ilfcolmargin}
    \linebreak

\vspace{\ilflinebreaksize}

\mbox{\Longstack{
      \ilfgerman{Jesus          } \cr
      \ilfgrammar{\enquote{Jesus}} \cr
      \ilfipa{ˈjeːzʊs        } \cr
      \ilfenglish{Jesus          } \cr
  }} \hspace{\ilfcolmargin}
  \mbox{\Longstack{
      \ilfgerman{und          } \cr
      \ilfgrammar{\enquote{und}} \cr
      \ilfipa{ʊnt          } \cr
      \ilfenglish{and          } \cr
  }} \hspace{\ilfcolmargin}
  \mbox{\Longstack{
      \ilfgerman{Maria!} \cr
      \ilfgrammar{\enquote{Maria}} \cr
      \ilfipa{maˈʁiːa } \cr
      \ilfenglish{Mary!} \cr
  }} \hspace{\ilfcolmargin}
    \linebreak

\vspace{\ilflinebreaksize}

\mbox{\Longstack{
      \ilfgerman{Was  } \cr
      \ilfgrammar{\enquote{was}} \cr
      \ilfipa{vas  } \cr
      \ilfenglish{What } \cr
  }} \hspace{\ilfcolmargin}
  \mbox{\Longstack{
      \ilfgerman{trug                             } \cr
      \ilfgrammar{3. Pers. Prät. Akt. Ind. \enquote{tragen}} \cr
      \ilfipa{tʁuːk                            } \cr
      \ilfenglish{did carry                        } \cr
  }} \hspace{\ilfcolmargin}
  \mbox{\Longstack{
      \ilfgerman{Maria            } \cr
      \ilfgrammar{Nom. Sgl. \enquote{Maria}} \cr
      \ilfipa{maˈʁiːa          } \cr
      \ilfenglish{Mary             } \cr
  }} \hspace{\ilfcolmargin}
  \mbox{\Longstack{
      \ilfgerman{unter  } \cr
      \ilfgrammar{\enquote{unter}} \cr
      \ilfipa{ˈʊntɐ  } \cr
      \ilfenglish{under  } \cr
  }} \hspace{\ilfcolmargin}
  \mbox{\Longstack{
      \ilfgerman{ihrem                            } \cr
      \ilfgrammar{Dat. Sgl. Poss. Pronomen 3. Pers.} \cr
      \ilfipa{ˈiːʁəm                           } \cr
      \ilfenglish{her                              } \cr
  }} \hspace{\ilfcolmargin}
  \mbox{\Longstack{
      \ilfgerman{Herzen?} \cr
      \ilfgrammar{Dat. Sgl. \enquote{Herz}} \cr
      \ilfipa{ˈhɛʁt͡sn̩} \cr
      \ilfenglish{heart?} \cr
  }} \hspace{\ilfcolmargin}
    \linebreak

\vspace{\ilflinebreaksize}

\mbox{\Longstack{
      \ilfgerman{Kyrieleison!} \cr
      \ilfgrammar{\enquote{Kyrieleison}} \cr
      \ilfipa{ˌkyːʁiː ʔeˈlɛɪ̯zɔn} \cr
      \ilfenglish{Kyrieleison!} \cr
  }} \hspace{\ilfcolmargin}
    \linebreak

\vspace{\ilflinebreaksize}

\mbox{\Longstack{
      \ilfgerman{ein          } \cr
      \ilfgrammar{aɪ̯n         } \cr
      \ilfipa{\enquote{ein}} \cr
      \ilfenglish{a            } \cr
  }} \hspace{\ilfcolmargin}
  \mbox{\Longstack{
      \ilfgerman{kleines               } \cr
      \ilfgrammar{ˈklaɪ̯nəs             } \cr
      \ilfipa{Neutr. \enquote{klein}} \cr
      \ilfenglish{little                } \cr
  }} \hspace{\ilfcolmargin}
  \mbox{\Longstack{
      \ilfgerman{Kindlein           } \cr
      \ilfgrammar{kɪntlaɪ̯n          } \cr
      \ilfipa{Dim. \enquote{Kind}} \cr
      \ilfenglish{child              } \cr
  }} \hspace{\ilfcolmargin}
  \mbox{\Longstack{
      \ilfgerman{ohne          } \cr
      \ilfgrammar{ˈoːnə         } \cr
      \ilfipa{\enquote{ohne}} \cr
      \ilfenglish{without       } \cr
  }} \hspace{\ilfcolmargin}
  \mbox{\Longstack{
      \ilfgerman{Schmerzen,} \cr
      \ilfgrammar{ˈʃmɛʁt͡sn̩ } \cr
      \ilfipa{Akk. Pl. \enquote{Schmerz}} \cr
      \ilfenglish{any pain} \cr
  }} \hspace{\ilfcolmargin}
    \linebreak

\vspace{\ilflinebreaksize}

\mbox{\Longstack{
      \ilfgerman{das          } \cr
      \ilfgrammar{\enquote{das}} \cr
      \ilfipa{das          } \cr
      \ilfenglish{that         } \cr
  }} \hspace{\ilfcolmargin}
  \mbox{\Longstack{
      \ilfgerman{trug                                     } \cr
      \ilfgrammar{3. Pers. Prät. Akt. Ind. \enquote{tragen}} \cr
      \ilfipa{tʁuːk                                    } \cr
      \ilfenglish{carried                                  } \cr
  }} \hspace{\ilfcolmargin}
  \mbox{\Longstack{
      \ilfgerman{Maria                    } \cr
      \ilfgrammar{Nom. Sgl. \enquote{Maria}} \cr
      \ilfipa{maˈʁiːa                  } \cr
      \ilfenglish{Mary                     } \cr
  }} \hspace{\ilfcolmargin}
  \mbox{\Longstack{
      \ilfgerman{unter          } \cr
      \ilfgrammar{\enquote{unter}} \cr
      \ilfipa{ˈʊntɐ          } \cr
      \ilfenglish{under          } \cr
  }} \hspace{\ilfcolmargin}
  \mbox{\Longstack{
      \ilfgerman{ihrem                              } \cr
      \ilfgrammar{Dat. Sgl. Poss.-Pron. \enquote{ihr}} \cr
      \ilfipa{ˈiːʁəm                             } \cr
      \ilfenglish{her                                } \cr
  }} \hspace{\ilfcolmargin}
  \mbox{\Longstack{
      \ilfgerman{Herzen.} \cr
      \ilfgrammar{Dat. Sgl. \enquote{Herz}} \cr
      \ilfipa{ˈhɛʁt͡sn̩} \cr
      \ilfenglish{heart} \cr
  }} \hspace{\ilfcolmargin}
    \linebreak

\vspace{\ilflinebreaksize}

\mbox{\Longstack{
      \ilfgerman{Jesus          } \cr
      \ilfgrammar{\enquote{Jesus}} \cr
      \ilfipa{ˈjeːzʊs        } \cr
      \ilfenglish{Jesus          } \cr
  }} \hspace{\ilfcolmargin}
  \mbox{\Longstack{
      \ilfgerman{und          } \cr
      \ilfgrammar{\enquote{und}} \cr
      \ilfipa{ʊnt          } \cr
      \ilfenglish{and          } \cr
  }} \hspace{\ilfcolmargin}
  \mbox{\Longstack{
      \ilfgerman{Maria!} \cr
      \ilfgrammar{\enquote{Maria}} \cr
      \ilfipa{maˈʁiːa} \cr
      \ilfenglish{Mary} \cr
  }} \hspace{\ilfcolmargin}
    \linebreak

\vspace{\ilflinebreaksize}

\mbox{\Longstack{
      \ilfgerman{Da          } \cr
      \ilfgrammar{\enquote{da}} \cr
      \ilfipa{daː         } \cr
      \ilfenglish{There       } \cr
  }} \hspace{\ilfcolmargin}
  \mbox{\Longstack{
      \ilfgerman{haben                                       } \cr
      \ilfgrammar{3. Pers. Pl. Präs. Akt. Ind. \enquote{haben}} \cr
      \ilfipa{ˈhaːbn̩                                     } \cr
      \ilfenglish{have                                        } \cr
  }} \hspace{\ilfcolmargin}
  \mbox{\Longstack{
      \ilfgerman{die          } \cr
      \ilfgrammar{\enquote{die}} \cr
      \ilfipa{diː          } \cr
      \ilfenglish{the          } \cr
  }} \hspace{\ilfcolmargin}
  \mbox{\Longstack{
      \ilfgerman{Dornen                 } \cr
      \ilfgrammar{Nom. Pl. \enquote{Dorn}} \cr
      \ilfipa{ˈdɔʁnən                } \cr
      \ilfenglish{thorns                 } \cr
  }} \hspace{\ilfcolmargin}
  \mbox{\Longstack{
      \ilfgerman{Rosen                  } \cr
      \ilfgrammar{Akk. Pl. \enquote{Rose}} \cr
      \ilfipa{ˈʁoːzn̩                } \cr
      \ilfenglish{roses                  } \cr
  }} \hspace{\ilfcolmargin}
  \mbox{\Longstack{
      \ilfgerman{getrag’n!} \cr
      \ilfgrammar{Part. Perf. \enquote{tragen}} \cr
      \ilfipa{ɡəˈtʁaːɡn̩} \cr
      \ilfenglish{beared} \cr
  }} \hspace{\ilfcolmargin}
    \linebreak

\vspace{\ilflinebreaksize}

\mbox{\Longstack{
      \ilfgerman{Kyrieleison!} \cr
      \ilfgrammar{\enquote{Kyrieleison}} \cr
      \ilfipa{ˌkyːʁiː ʔeˈlɛɪ̯zɔn} \cr
      \ilfenglish{Kyrieleison!} \cr
  }} \hspace{\ilfcolmargin}
    \linebreak

\vspace{\ilflinebreaksize}

\end{flushleft}

\subsection{Abkürzungen}
\begin{multicols}{4}
  \noindent Abl.: \textit{Ablativ} \\
  Akk.: \textit{Akkusativ} \\
  Akt.: \textit{Aktiv} \\
  Dat.: \textit{Dativ} \\
  Dim.: \textit{Diminutiv} \\
  Ind.: \textit{Indikativ} \\
  Nom.: \textit{Nominativ} \\
  Part.: \textit{Partizip} \\
  Perf.: \textit{Perfekt} \\
  Pers.: \textit{Person} \\
  Pl.: \textit{Plural} \\
  Präs.: \textit{Präsens} \\
  Prät.: \textit{Präteritum} \\
  Sgl.: \textit{Singular} \\
\end{multicols} 


\end{document}